\definecolor{Green}{rgb}{0.0, 0.5, 0.0}
\definecolor{Red}{rgb}{0.7, 0.0, 0.0}
\definecolor{Yellow}{rgb}{1.0, 1.0, 0.2}



\label{sec:technologievorstellung}
Webanwendungen stellen eine Zusammensetzung aus verschiedenen Technologien, Frameworks und Architekturmustern dar. Die Vielzahl an verfügbaren Technologien bietet dabei einen breiten Lösungsraum für ähnliche Systeme, die verschiedene Technologien nutzen, aber die gleichen Anforderungen erfüllen. Die Stärken sowie Schwächen der Systeme, die für die Zukunftsfähigkeit des Systems essentiell ist, wird jedoch erst bei näheren Betrachtung deutlich .
Daher ist es für den Prototypen besonders wichtig vor der Konzeptionsphase einen Überblick über die unterschiedlichen Technologien sowie Architekturmuster zu schaffen um in der anschließend Konzeptionsphase eine Diskussion der Vor- und Nachteile zu ermöglichen um darauf aufbauend eine begründbare als auch nachvollziehbare Entscheidung für die genutzten Technologie zu ermöglichen.
TODO: Ich denke zu jeder Technologie würde ich gerne vorstellen welche große Firmen diese nutzt. Sinn dahinter ist es sozusagen die Skalierbarkeit zu zeigen. In unserem Sinne: Ja große Firmen nutzen diese, als klappts auch für den Prototypen und Skalierung ist prinzipiell unbdenklich


\section{Architekturstyles}
Architekturstil setzt das Vokabular fest und ist mit verantwortlich für die größten Constraints des Systems.
\subsection{Monolith}
\subsection{Microarchitecture}
\subsection{REST}


\section{Datenbanken}

\subsection{Relationale Datenbanken}
\subsubsection{MariaDB und MySQL}
\subsubsection{PostgreSQL}

\subsection{NoSql Datenbanken}
todo: rede und erläutere die unterschiedlichen Arten ()ocument storage, Graphendatenbanken, key value, spaltenorientert).
Dann rede, dass für das System Documenten storage eher von bedeutung ist.
\subsubsection{MongoDb}
\subsubsection{ggf. noch eine weitere? Eig. nicht nötig, da ja alle ähnlich}

\section{Kommunikation}
Rede warum Kommunikation wichtig ist und erläutere die unterschiedlichen Arten für die Kommunikation. Erwähne dabei ruhig inter und outer Kommunikation. Aber stelle diese nicht als eigene subsection vor, weil viele ja sowohl frontend(outer) als auch backend(inter) genutzt werden können.


\subsubsection{gRPC}
\subsubsection{Advanced Message Queuing Protocol}
\subsubsection{RabbitMQ}
\subsubsection{Kafka}



\subsubsection{HTTP}

\subsubsection{Client Polling}

\subsubsection{Websockets}

\subsubsection{Server Sent Events}
